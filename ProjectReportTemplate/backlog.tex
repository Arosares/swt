
\subsection*{Stories completed in Sprint 1}

\begin{table}[h]
  \caption{User Story 12: Store XML in classes, Front}
  \label{Story_12_Front}
  \centering
  \begin{tabular}{|p{9cm} p{2cm}|}
	\hline  	
  	Store XML in classes & Nr. 12 \\ 
  	\hline
  	As a developer, &    \\ 
  	I want to store the XML test input in workable classes so that I don't have to parse the XML every time I want to analyse old test runs and I can easily compare different test runs. &    \\ 
  	Size: M & Sprint: 1 \\ 
  	\hline
  \end{tabular}
\end{table}
\begin{table}[h]
  \caption{User Story 12: Store XML in classes, Back}
  \label{Story_12_Back}
  \centering
  \begin{tabular}{|p{10cm} p{1cm}|}
  \hline
  	  &    \\ 
  	The story is done when &    \\ 
  	 - appropriate classes are created for the parsed information are implemented & \\
  	 - all information given in the XML input is extracted into classes &    \\ 
  	 
  	  &  
  	   \\ 
  	\hline
  \end{tabular}
\end{table}

\subsection*{Stories completed in Sprint 2}

\begin{table}[H]
  \caption{User Story 18: XML Parsing, Front}
  \label{Story_18_Front}
  \centering
  \begin{tabular}{|p{9cm} p{2cm}|}
	\hline  	
  	XML Parsing & Nr. 18 \\ 
  	\hline
  	As a medatixx tester, &    \\ 
  	I want the TDA to parse the provided XML test file format so that I don't have to adopt the test run's output.  &    \\ 
  	Size: M & Sprint: 1 \\ 
  	\hline
  \end{tabular}
\end{table}
\begin{table}[H]
  \caption{User Story 18: XML Parsing, Back}
  \label{Story_18_Back}
  \centering
  \begin{tabular}{|p{10cm} p{1cm}|}
  \hline
  	  &    \\ 
  	The story is done when &    \\ 
  	 - the StAX parser is able to extract the relevant information from the XML test files & \\ 
  	 
  	  &  
  	   \\ 
  	\hline
  \end{tabular}
\end{table}

\ \\ 

\begin{table}[H]
  \caption{User Story 2: Show Imports in Tree Structure, Front}
  \label{Story_2_Front}
  \centering
  \begin{tabular}{|p{9cm} p{2cm}|}
	\hline  	
  	Show Imports in Tree Structure & Nr. 2 \\ 
  	\hline
  	As a user, &    \\ 
  	I want to see all loaded test runs in a tree structure so that I can see where the corresponding XML files lie in the file system. &    \\ 
  	Size: M & Sprint: 2 \\ 
  	\hline
  \end{tabular}
\end{table}
\begin{table}[H]
  \caption{User Story 2: Show Imports in Tree Structure, Back}
  \label{Story_2_Back}
  \centering
  \begin{tabular}{|p{10cm} p{1cm}|}
  \hline
  	  &    \\ 
  	The story is done when &    \\ 
  	 - the parsed XML files are recorded as tree items & \\ 
  	 - the tree items are displayed in a tree view in the GUI & \\ 
  	 - the tree view is displayed according to the location of the files in the file system & \\ 
  	 - the tree items can be clicked and the corresponding test run table is displayed & \\ 
  	 
  	  &  
  	   \\ 
  	\hline
  \end{tabular}
\end{table}

\ \\ 

\begin{table}[H]
  \caption{User Story 9: Test Run Selection, Front}
  \label{Story_9_Front}
  \centering
  \begin{tabular}{|p{9cm} p{2cm}|}
	\hline  	
  	Test Run Selection & Nr. 9 \\ 
  	\hline
  	As a user, &    \\ 
  	I want to be able to select a specific test run so that I can get information about it. &    \\ 
  	Size: M & Sprint: 1 \\ 
  	\hline
  \end{tabular}
\end{table}
\begin{table}[H]
  \caption{User Story 9: Test Run Selection, Back}
  \label{Story_2_Back}
  \centering
  \begin{tabular}{|p{10cm} p{1cm}|}
  \hline
  	  &    \\ 
  	The story is done when &    \\ 
  	 - one or more XML files can be selected and parsed & \\ 
  	 - a root folder can be selected and all XML files in it and its subfolders are parsed & \\  
  	 
  	  &  
  	   \\ 
  	\hline
  \end{tabular}
\end{table}

\subsection*{Stories completed in Sprint 3}

\begin{table}[H]
  \caption{User Story 7: Test Run Table, Front}
  \label{Story_7_Front}
  \centering
  \begin{tabular}{|p{9cm} p{2cm}|}
	\hline  	
  	Test Run Table & Nr. 7 \\ 
  	\hline
  	As a user, &    \\ 
  	I want to display a table containing all the classes of one specific test run and their failure percentage so that I can get a quick overview of the most problematic classes. &    \\ 
  	Size: M & Sprint: 2 \\ 
  	\hline
  \end{tabular}
\end{table}
\begin{table}[H]
  \caption{User Story 7: Test Run Table, Back}
  \label{Story_7_Back}
  \centering
  \begin{tabular}{|p{10cm} p{1cm}|}
  \hline
  	  &    \\ 
  	The story is done when &    \\ 
  	 - a table is displayed in the view & \\ 
  	 - the table is filled with the classes of a test run and their failure percentages & \\ 
  	 - the classes are sorted in descending order of failure percentage & \\ 
  	 - the classes with the highest failure percentage are highlighted & \\  
  	 
  	  &  
  	   \\ 
  	\hline
  \end{tabular}
\end{table}

\ \\ 

\begin{table}[H]
  \caption{User Story 3: Test Run Chart, Front}
  \label{Story_3_Front}
  \centering
  \begin{tabular}{|p{9cm} p{2cm}|}
	\hline  	
  	Test Run Chart & Nr. 3 \\ 
  	\hline
  	As a user, &    \\ 
  	I want to display a chart that shows the failure percentages of a specific class over all loaded test runs so that I can see the evolution of the failure percentage. &    \\ 
  	Size: M & Sprint: 3 \\ 
  	\hline
  \end{tabular}
\end{table}
\begin{table}[H]
  \caption{User Story 3: Test Run Chart, Back}
  \label{Story_3_Back}
  \centering
  \begin{tabular}{|p{10cm} p{1cm}|}
  \hline
  	  &    \\ 
  	The story is done when &    \\ 
  	 - the chart is displayed in the GUI & \\ 
  	 - the chart is filled with the class' failure percentages of all loaded test runs & \\ 
  	 - the chart is ordered by the starting times of the test runs & \\ 

  	 
  	  &  
  	   \\ 
  	\hline
  \end{tabular}
\end{table}

\subsection*{Stories completed in Sprint 4}

No user stories were completed in sprint 4

\subsection*{Stories completed in Sprint 5}


\begin{table}[H]
  \caption{User Story 23: Implement Apriori Algorithm, Front}
  \label{Story_23_Front}
  \centering
  \begin{tabular}{|p{9cm} p{2cm}|}
	\hline  	
  	Implement Apriori Algorithm & Nr. 23 \\ 
  	\hline
  	As a developer, &    \\ 
  	I want to implement the Apriori algorithm so that I can generate the frequent item sets and the strong rules of loaded test runs. &    \\ 
  	Size: L & Sprint: 4 \\ 
  	\hline
  \end{tabular}
\end{table}
\begin{table}[H]
  \caption{User Story 23: Implement Apriori Algorithm, Back}
  \label{Story_23_Back}
  \centering
  \begin{tabular}{|p{10cm} p{1cm}|}
  \hline
  	  &    \\ 
  	The story is done when &    \\ 
  	 - the frequent item sets are calculated & \\ 
  	 - the strong rules are generated from the item sets & \\ 
  	 - filtering for confidence is possible & \\ 
  	 - filtering for class distance is possible & \\ 
  	 
  	  &  
  	   \\ 
  	\hline
  \end{tabular}
\end{table}

\ \\ 

\begin{table}[H]
  \caption{User Story 24: Display Analysis Data, Front}
  \label{Story_24_Front}
  \centering
  \begin{tabular}{|p{9cm} p{2cm}|}
	\hline  	
  	Display Analysis Data & Nr. 24 \\ 
  	\hline
  	As a user, &    \\ 
  	I want to view the results of the Apriori algorithm in a graphically enriched view so that I immediately see the dependencies between failed classes. &    \\ 
  	Size: L & Sprint: 4 \\ 
  	\hline
  \end{tabular}
\end{table}
\begin{table}[H]
  \caption{User Story 24: Display Analysis Data, Back}
  \label{Story_24_Back}
  \centering
  \begin{tabular}{|p{10cm} p{1cm}|}
  \hline
  	  &    \\ 
  	The story is done when &    \\ 
  	 - the the interface to the Apriori algorithm is implemented & \\ 
  	 - a mock up is drawn and accepted by the client & \\ 
  	 - the design is layouted using JavaFX & \\ 
  	 - the Apriori results are displayed in the approved layout & \\ 
  	 
  	  &  
  	   \\ 
  	\hline
  \end{tabular}
\end{table}

\ \\ 

\begin{table}[H]
  \caption{User Story 213: Evolution of a class, Front}
  \label{Story_213_Front}
  \centering
  \begin{tabular}{|p{9cm} p{2cm}|}
	\hline  	
  	Association Analysis & Nr. 8 \\ 
  	\hline
  	As a user, &    \\ 
  	I want to see the differences in unit tests of a specific class in two different test runs so that I know what changed (improvement / deterioration) &    \\ 
  	Size: M & Sprint: 4 \\ 
  	\hline
  \end{tabular}
\end{table}
\begin{table}[H]
  \caption{User Story 213: Evolution of a class, Back}
  \label{Story_213_Back}
  \centering
  \begin{tabular}{|p{10cm} p{1cm}|}
  \hline
  	  &    \\ 
  	The story is done when &    \\ 
  	 - data points can be selected in the test run chart & \\ 
  	 - the differences between two data points are calculated & \\ 
  	 - the comparison results are displayed in the GUI & \\  
  	 
  	  &  
  	   \\ 
  	\hline
  \end{tabular}
\end{table}

\ \\

\begin{table}[H]
  \caption{User Story 28: Creating a help function, Front}
  \label{Story_28_Front}
  \centering
  \begin{tabular}{|p{9cm} p{2cm}|}
	\hline  	
  	Creating a help function & Nr. 28 \\ 
  	\hline
  	As a user, &    \\ 
  	I want to be able to read a manual so that I know all the features and how the program works. &    \\ 
  	Size: S & Sprint: 5 \\ 
  	\hline
  \end{tabular}
\end{table}
\begin{table}[H]
  \caption{User Story 28: Creating a help function, Back}
  \label{Story_28_Back}
  \centering
  \begin{tabular}{|p{10cm} p{1cm}|}
  \hline
  	  &    \\ 
  	The story is done when &    \\ 
  	 - the manual is written & \\ 
  	 - the manual is included in the TDA & \\ 
  	 - a button is accessible in the GUI showing the manual & \\
  	 
  	  &  
  	   \\ 
  	\hline
  \end{tabular}
\end{table} 

\ \\ 

\begin{table}[H]
  \caption{User Story 27: Distribution with suited license, Front}
  \label{Story_27_Front}
  \centering
  \begin{tabular}{|p{9cm} p{2cm}|}
	\hline  	
  	Distribution with suited license & Nr. 27 \\ 
  	\hline
  	As a client, &    \\ 
  	I want the software to be released using a suited license and having it displayed in the program so that everyone can see which license got used. &    \\ 
  	Size: M & Sprint: 5 \\ 
  	\hline
  \end{tabular}
\end{table}
\begin{table}[H]
  \caption{User Story 27: Distribution with suited license, Back}
  \label{Story_27_Back}
  \centering
  \begin{tabular}{|p{10cm} p{1cm}|}
  \hline
  	  &    \\ 
  	The story is done when &    \\ 
  	 - a fitting license is chosen & \\ 
  	 - the chosen license is used for TDA & \\ 
  	 - a button is visible in the GUI for displaying the license & \\  
  	 
  	  &  
  	   \\ 
  	\hline
  \end{tabular}
\end{table} 

\ \\ 

\begin{table}[H]
  \caption{User Story 21: Documentation, Front}
  \label{Story_21_Front}
  \centering
  \begin{tabular}{|p{9cm} p{2cm}|}
	\hline  	
  	Documentation & Nr. 21 \\ 
  	\hline
  	As Lehrstuhl SWT, &    \\ 
  	I want to see detailed documentation so that I can follow the group's progress. &    \\ 
  	Size: L & Sprint: 5 \\ 
  	\hline
  \end{tabular}
\end{table}
\begin{table}[H]
  \caption{User Story 21: Documentation, Back}
  \label{Story_21_Back}
  \centering
  \begin{tabular}{|p{10cm} p{1cm}|}
  \hline
  	  &    \\ 
  	The story is done when &    \\ 
  	 - the meetings of all Sprints are documented in the group's wiki in the vc & \\ 
  	 
  	  &  
  	   \\ 
  	\hline
  \end{tabular}
\end{table} 

\subsection*{Not completed Stories}

\begin{table}[H]
  \caption{User Story 1: Testing, Front}
  \label{Story_1_Front}
  \centering
  \begin{tabular}{|p{9cm} p{2cm}|}
	\hline  	
  	Testing & Nr. 1 \\ 
  	\hline
  	As a client, &    \\ 
  	I want the software to be tested so that it runs reliably. &    \\ 
  	Size: L & Sprint: 5 \\ 
  	\hline
  \end{tabular}
\end{table}
\begin{table}[H]
  \caption{User Story 1: Testing, Back}
  \label{Story_1_Back}
  \centering
  \begin{tabular}{|p{10cm} p{1cm}|}
  \hline
  	  &    \\ 
  	The story is done when &    \\ 
  	 - all classes and functions are tested & \\ 
  	 - the tests have passed & \\ 
  	 
  	  &  
  	   \\ 
  	\hline
  \end{tabular}
\end{table} 

\ \\ 

\begin{table}[H]
  \caption{User Story 25: Additional usage scenario, Front}
  \label{Story_25_Front}
  \centering
  \begin{tabular}{|p{9cm} p{2cm}|}
	\hline  	
  	Additional usage scenario & Nr. 25 \\ 
  	\hline
  	As a client, &    \\ 
  	I want to have an additional, valuable feature so that I can further analyse the test runs.  &    \\ 
  	Size: M & Sprint: 5 \\ 
  	\hline
  \end{tabular}
\end{table}
\begin{table}[H]
  \caption{User Story 25: Additional usage scenario, Back}
  \label{Story_25_Back}
  \centering
  \begin{tabular}{|p{10cm} p{1cm}|}
  \hline
  	  &    \\ 
  	The story is done when &    \\ 
  	 - an additional usage scenario is discovered & \\ 
  	 - the scenario is accepted by the client and the Lehrstuhl & \\ 
  	 - the scenario is implemented & \\ 
  	 
  	  &  
  	   \\ 
  	\hline
  \end{tabular}
\end{table}  

\ \\ 

\begin{table}[H]
  \caption{User Story 26: Code Documentation, Front}
  \label{Story_26_Front}
  \centering
  \begin{tabular}{|p{9cm} p{2cm}|}
	\hline  	
  	Code Documentation & Nr. 26 \\ 
  	\hline
  	As a developer, &    \\ 
  	I want well documented code, so that it is easily readable and understandable. &    \\ 
  	Size: S & Sprint: 5 \\ 
  	\hline
  \end{tabular}
\end{table}
\begin{table}[H]
  \caption{User Story 26: Code Documentation, Back}
  \label{Story_26_Back}
  \centering
  \begin{tabular}{|p{10cm} p{1cm}|}
  \hline
  	  &    \\ 
  	The story is done when &    \\ 
  	 - All classes and methods are documented with Java doc & \\ 
  	 
  	  &  
  	   \\ 
  	\hline
  \end{tabular}
\end{table} 

\subsection*{Other Stories}

\begin{table}[H]
  \caption{User Story 8: Association Analysis, Front}
  \label{Story_8_Front}
  \centering
  \begin{tabular}{|p{9cm} p{2cm}|}
	\hline  	
  	Association Analysis & Nr. 8 \\ 
  	\hline
  	As a user, &    \\ 
  	I want to see possible relations between the failure percentages of different classes so that I can learn about dependencies between failed classes. &    \\ 
  	Size: L & Sprint: 3 \\ 
  	\hline
  \end{tabular}
\end{table}
\begin{table}[H]
  \caption{User Story 8: Association Analysis, Back}
  \label{Story_8_Back}
  \centering
  \begin{tabular}{|p{10cm} p{1cm}|}
  \hline
  	  &    \\ 
  	The story is done when &    \\ 
  	 - the frequent item sets are calculated & \\ 
  	 - the strong rules are generated from the item sets & \\ 
  	 - filtering for confidence is possible & \\ 
  	 - filtering for class distance is possible & \\ 
  	 - the Apriori algorithm is implemented and working & \\ 
  	 - the Apriori algorithms results are displayed in the GUI & \\ 
  	 
  	  &  
  	   \\ 
  	\hline
  \end{tabular}
\end{table} 

\ \\ 

\subsection*{Research Stories} 

\begin{table}[H]
  \caption{Research Story R1: Research Apriori, Front}
  \label{Story_R1_Front}
  \centering
  \begin{tabular}{|p{9cm} p{2cm}|}
	\hline  	
  	Research Apriori & Nr. R1 \\ 
  	\hline
  	We wish to understand, &    \\ 
  	how the Apriori algorithm works with the help of scientific research papers and the internet. &    \\ 
  	Size: M & Sprint: 1 \\ 
  	\hline
  \end{tabular}
\end{table}
\begin{table}[H]
  \caption{Research Story R1: Research Apriori, Back}
  \label{Story_R1_Back}
  \centering
  \begin{tabular}{|p{4cm} p{7cm}|}
  \hline 
 	Origin & Project brief  \\ 
  	\hline
  	Duration & minimum 1 day \\ 
  	 & maximum 2 days \\ 
  	 
  	  &  
  	   \\ 
  	\hline
  \end{tabular}
\end{table} 

\ \\

\begin{table}[H]
  \caption{Research Story R2: Research StAX Parser, Front}
  \label{Story_R2_Front}
  \centering
  \begin{tabular}{|p{9cm} p{2cm}|}
	\hline  	
  	Research StAX Parser & Nr. R2 \\ 
  	\hline
  	We wish to understand, &    \\ 
  	how the StAX parser works and is used for parsing the given XML structure with the help of its official documentation and other tutorials. &    \\ 
  	Size: S & Sprint: 1 \\ 
  	\hline
  \end{tabular}
\end{table}
\begin{table}[H]
  \caption{Research Story R2: Research StAX Parser, Back}
  \label{Story_R2_Back}
  \centering
  \begin{tabular}{|p{4cm} p{7cm}|}
  \hline 
 	Origin & Project brief  \\ 
  	\hline
  	Duration & minimum 0.5 day \\ 
  	 & maximum 1 day \\ 
  	 
  	  &  
  	   \\ 
  	\hline
  \end{tabular}
\end{table} 

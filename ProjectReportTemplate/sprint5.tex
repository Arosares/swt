
\subsection{Sprint No.~5}


\subsubsection*{Sprint Planning}

Goal: Finalize functionality (Comparison View and Distance-Filter in Apriori) + Testing.
Deliver the idea of an additional Usage Scenario.
Make the product ready for Version 1.0 release.

The primary goal of Sprint 5 was to finish, test, and document everything in the project. This included writing tests where coverage wasn't sufficient, and improving code documentation which had lagged behind the occasional refactoring. For this purpose, we created an extra story for documentation and tasks for testing and documenting all sections of code we had.

\begin{itemize}
	\item User Story 1: Testing (Prio 1 / Size M)
	\end{itemize}
As a client,
I want the software to be tested,
such that it runs reliably.
\begin{table}[h]
  \caption{User Story 1 Remaining Tasks}
  \label{Story 1 Tasks}
  \centering
  \begin{tabular}{p{1cm}|p{5cm}|p{3cm}|}
  	Nr & Description & Assigned \\ 
  	\hline
  	1.1 & Write Logic tests (excl. Parser/Apriori) & Andreas Köllner \\ 
  	\hline
  	1.2 & Write Parser tests & Frank Keßler \\ 
  	\hline
  	1.3 & Write Apriori tests & Tobias Schwartz \\ 
  	\hline
  	1.4 & Write Controller tests &  Jan Martin \\ 
  	\hline
  	1.5 & Write View tests &  tbd \\ 
  	\hline
  	1.6 & Write Model tests &  tbd \\ 
  	\hline
  \end{tabular}
\end{table}

\begin{itemize}
	\item User Story 25: Additional Usage Scenario (Prio 2 / Size ?)
	\end{itemize}
As a client, I want a valuable feature, such that I can further analyze my testruns.
\begin{table}[h]
  \caption{User Story 25 Tasks}
  \label{Story 23 Tasks}
  \centering
  \begin{tabular}{p{1cm}|p{5cm}|p{3cm}|}
  	Nr & Description & Assigned \\ 
  	\hline
  	25.1 & Think about new feature & all \\ 
  	\hline
  	25.2 & Design Paper Mock-up & all \\ 
  	\hline
  	25.3 & Implement Mock-up & unassigned \\ 
  	\hline
  \end{tabular}
\end{table}

\begin{itemize}
	\item Remaining Tasks from Sprint 4
	\end{itemize}

\begin{table}[h]
  \caption{Remaining Sprint 4 Tasks}
  \label{Remaing Tasks}
  \centering
  \begin{tabular}{p{1cm}|p{5cm}|p{3cm}|}
  	Nr & Description  \\ 
  	\hline
  	213.3 & Implement the hovering over a datapoint in the chart. \\ 
  	\hline
  	213.8 & fill comparison/"evolution of a class" view with data \\ 
  	\hline
  	24.3 & Test: Display of Apriori outputs. \\ 
  	\hline
  	23.3 & Test; Max Tree Depth for View \\ 
  	\hline
  \end{tabular}
\end{table}
\begin{itemize}
	\item User Story 26: Code Documentation (Prio 2 / Size S)
	\end{itemize}
\begin{table}[h]
  \caption{User Story 26 Tasks}
  \label{Story 26 Tasks}
  \centering
  \begin{tabular}{p{1cm}|p{5cm}|p{3cm}|}
  	Nr & Description & Assigned \\ 
  	\hline
  	26.1 & Parser Documentation & Frank Keßler \\ 
  	\hline
  	23.2 & Apriori Documentation & Tobias Schwartz \\ 
  	\hline
  	23.3 & Database & Frank Keßler \\ 
  	\hline
  	23.4 & Controller &  Jan Martin \\ 
  	\hline
  	26.5 & View & Simon Meyer \\ 
  	\hline
  	26.6 & Model & Andreas Köllner \\ 
  	\hline
  	26.7 & TestedClass + UnitTest classes & Andreas Köllner \\ 
  	\hline
  	26.8 & TestRun, TestRunStartTime, Counters classes & Andreas Köllner \\ 
  	\hline
  	26.9 & TDAChart, TDAClassView classes & tbd \\ 
  	\hline
  	26.10 & TDAAnalyzerView, TDATable classes & tbd \\ 
  	\hline
  	26.11 & Menubar, TestRunTotals & tbd \\ 
  	\hline
  	26.12 & TreeView, RunComparison & tbd \\ 
  	\hline
  	26.13 & TreeNode & tbd \\ 
  	\hline
  \end{tabular}
\end{table}

\begin{itemize}
	\item User Story 27: x (Prio 1 / Size M)
	\end{itemize}
As a client,
I want the Software to be released using a suited License and having it displayed in the program,
such that everyone can see which license got used.	
\begin{table}[h]
  \caption{User Story 27 Tasks}
  \label{Story 23 Tasks}
  \centering
  \begin{tabular}{p{1cm}|p{5cm}|p{3cm}|}
  	Nr & Description & Assigned \\ 
  	\hline
  	27.1 & Look for best suited license & all \\ 
  	\hline
  	27.2 & Create button in menubar to view license information & tbd \\ 
  	\hline
  \end{tabular}
\end{table}

\newpage
\begin{itemize}
	\item User Story 28: x (Prio 1 / Size S)
	\end{itemize}
As a user,
I want to be able to read a manual
such that I know all the features and how the program works
\begin{table}[h]
  \caption{User Story 28 Tasks}
  \label{Story 28 Tasks}
  \centering
  \begin{tabular}{p{1cm}|p{5cm}|p{3cm}|}
  	Nr & Description & Assigned \\ 
  	\hline
  	28.1 & Write Manual for all functions & WIP \\ 
  	\hline
  	28.2 & Create Button and View for the Manual & tbd \\ 
  	\hline
  \end{tabular}
\end{table}

\subsubsection*{Noteworthy Development Aspects}

The Sprint was begun with taking stock of what we had. All Must-Have features were implemented and mostly functional.
Our primary aim for this last sprint was to thus wrap up development as cleanly as possible. 
As such, our focus was on ensuring the project was properly documented and tested, and our tasks and priorities reflected as much. 

As we did not manage to create a second additional usage scenario by sprint 4, and would have been unable to present one before implementation anyways, we didn't plan with it's implementation at all and contented ourselves with creating an additional scenario concept for presentation at the end of sprint 5. 
Thus, our only remaining features to be implemented were some leftover tasks from Sprint 4.
Documentation and testing progressed at a satisfactory pace, and helped us find some minor bugs and catch some exceptions that previously went unnoticed. We also found time to slightly improve the GUI, including better scaling for lower resolution screens.

\subsubsection*{Sprint Review}

We succeeded in closing any remaining tasks from Sprint 4, and significantly increased code documentation. 
Our first 'Additional Usage Scenario' was successfully implemented and tested, and we notably increased our test coverage, finding and fixing several bugs in the process. 
Sadly, we didn't manage to write documentation for all classes, nor increase our coverage to anywhere near 100%. 

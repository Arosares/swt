
\subsection{Sprint No.~1}

\emph{(Approx.~2--3~pages of text.) Andy}

\subsubsection*{Sprint Planning}

\emph{State the goal of and the user stories chosen for this sprint (sprint backlog). Detail the tasks that your group derived from each user story, and provide the names of the team members allocated to each task.} \\

Our goal for Sprint 1 was to implement the StAX parser to offer the basic functionality of parsing the given test run XML files into our internal data structure to the client. \\ 
We also wanted to design a first paper prototype of our system, so that we could get a first feedback on our design desicions. \\ 
Our Sprint backlog for this Sprint contained three user stories: \# 18 XML Parsing, \# 9 Test Run Selection and \# 12 Store XML in classes. Additionally we created one research story for this Sprint, belonging to the StAX parser and its functionalities. \\ 
The tasks we derived from our user stories can be found in the following tables. \\ 

\begin{table}[h]
  \caption{User Story number 18: XML Parsing}
  \label{US_Parsing}
  \centering
  \begin{tabular}{p{1.5cm}|p{9cm}|p{3cm}|}
  	Task number & Task description & Assigned to \\ 
  	\hline
  	\hline
  	18.1 & Create package structure for project & Frank \\
  	\hline
  	18.2 & Create parser class & Frank \\ 
  	\hline
  	18.3 & Find elements in XML file & Tobias, Frank, Andreas \\ 
  	\hline
  	18.4 & Implement parsing functionality & Tobias, Frank \\ 
  	\hline
  	18.5 & Store parsed information in data structure & Tobias, Frank, Andreas \\ 
  	\hline
  \end{tabular}
\end{table} 

\begin{table}[h]
  \caption{User Story number 9: Test Run Selection}
  \label{US_Selection}
  \centering
  \begin{tabular}{p{1.5cm}|p{9cm}|p{3cm}|}
  	Task number & Task description & Assigned to \\ 
  	\hline
  	\hline
  	\hline
  \end{tabular}
\end{table} 

\begin{table}[h]
  \caption{User Story number 12: Store XML in classes}
  \label{US_Storage}
  \centering
  \begin{tabular}{p{1.5cm}|p{9cm}|p{3cm}|}
  	Task number & Task description & Assigned to \\ 
  	\hline
  	\hline
  	12.1 & Identify classes to represent information stored in XML files & all \\ 
  	\hline
  	12.2 & Create class TestData & Jan \\ 
  	\hline
  	12.3 & Create classes TestRun, TestedClass and UnitTest & Simon \\ 
  	\hline
  \end{tabular}
\end{table} 

\subsubsection*{Noteworthy Development Aspects}

\emph{Describe and justify the development approach taken and the artefacts produced in this sprint (e.g., prototypes).  State any peculiarities of this sprint, such as peculiarities  regarding (i)~adopted development practices, (ii)~encountered obstacles, (iii)~questions that arose and needed clarification possibly from the client, or (iv)~important aspects regarding --- or changes to --- your software architecture, your algorithms or your techniques applied to solve a technical problem.}

\subsubsection*{Sprint Review}

\emph{Describe the product increment produced in this sprint. Compare the achieved increment with the sprint goal and the user stories that were chosen for this sprint. Give a brief summary on your group's retrospective, including changes to the product backlog and also to the development process and/or techniques that you installed after the sprint in order to overcome any identified obstacle.}

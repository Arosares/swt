
\subsection{Sprint No.~4}

\emph{(Approx.~2--3~pages of text.)}

\subsubsection*{Sprint Planning}

Goal: Rework apriori algorithm and display its results and implement evolution of classes' unittest comparison (additional usage scenario)

The goal for Sprint 4 was to rework the Apriori Algorithm for higher performance, and actually display it's results in our GUI, as well as the implementation of our first bonus usage scenario, which we suggested in Sprint 1. 
For this purpose, we created a new User Story for Apriori in line with our effort to improve internal documentation.
We also started splitting User Stories into more tasks than before and introduced a general category for tasks that could be done if time allowed, mostly related to polishing or minor bug fixes.

\begin{itemize}
	\item User Story 23: Implement Apriori Algorithm (Prio 1 / Size L)
	\end{itemize}
As a developer,
I want to implement the apriori algorithm
so that I can generate the frequent item sets and the strong rules of loaded test runs
\begin{table}[h]
  \caption{User Story 23 Tasks}
  \label{Story 23 Tasks}
  \centering
  \begin{tabular}{p{1cm}|p{5cm}|p{3cm}|}
  	Nr & Description & Assigned \\ 
  	\hline
  	23.1 & Get failed classes of one test run & Tobias Schwartz, Jan Martin \\ 
  	\hline
  	23.2 & Generate frequent item sets & Tobias Schwartz \\ 
  	\hline
  	23.3 & Filter for class distances & Jan Martin \\ 
  	\hline
  	23.4 & Generate strong rules & Tobias Schwartz, Jan Martin \\ 
  	\hline
  \end{tabular}
\end{table}

\newpage
\begin{itemize}
	\item User Story 24: Display Analysis Data (Prio 2 / Size M)
	\end{itemize}
As a user,
I want to view the results of the apriori algorithm in a graphically enriched view
so that I immediately see the dependencies between failed classes.
\begin{table}[h]
  \caption{User Story 24 Tasks}
  \label{Story 24 Tasks}
  \centering
  \begin{tabular}{p{1cm}|p{5cm}|p{3cm}|}
  	Nr & Description & Assigned \\ 
  	\hline
  	24.1 & sketch mockup of display & Tobias Schwartz \\ 
  	\hline
  	24.2 & implement mockup as a layout in JavaFx & Tobias Schwartz \\ 
  	\hline
  	24.3 & fill apriori display with calculated data & Tobias Schwartz \\ 
  	\hline
  \end{tabular}
\end{table}

\begin{itemize}
	\item General Tasks (No User Story)
	\end{itemize}

\begin{table}[h]
  \caption{General Tasks}
  \label{General Tasks}
  \centering
  \begin{tabular}{p{1cm}|p{5cm}|p{3cm}|}
  	Nr & Description & Assigned \\ 
  	\hline
  	(Gen) & clear data: clear testrunsummary as well & --- \\ 
  	\hline
  	(Gen) & change tableview highlighting (background, textcolour) & --- \\ 
  	\hline
  	(Gen) & hide classes with 0 failure percentage & --- \\ 
  	\hline
  	(Gen) & Update UseCase- and Scope-of-Workarea Diagrams & Andreas Köllner \\ 
  	\hline
  	(Gen) & digitalize paper-mockups & Andreas Köllner \\ 
  	\hline
  \end{tabular}
\end{table}
We used the "General" category to sum up tasks that a teammember could do when their main task was done for the sprint increment or they otherwise had to wait on the work of another team member. 

\begin{itemize}
	\item User Story 213: Evolution of a class (additional usage scenario) (Prio 1 / Size L)
	\end{itemize}
As a user,
I want to see the differences in Unittests of a specific class in two selected test runs
so that I know what changed (improvement/deterioration).
\begin{table}[h]
  \caption{User Story 213 Tasks}
  \label{Story 213 Tasks}
  \centering
  \begin{tabular}{p{1cm}|p{5cm}|p{3cm}|}
  	Nr & Description & Assigned \\ 
  	\hline
  	213.1 & Mockup of additional usage scenario "evolution of a class" & Simon Meyer \\ 
  	\hline
  	213.2* & Change all doubleclicks to singleclicks & Frank Kessler \\ 
  	\hline
  	213.3 & Hover chart node for short infos (incl. timestamp) & Frank Kessler \\ 
  	\hline
  	213.4 & Chart in own Window & Frank Kessler \\ 
  	\hline
  	213.5 & Calculate failure percentage of one class between two testruns & Frank Kessler \\ 
  	\hline
  	213.6 & implement comparison layout according to previous mockup and feedback & Simon Meyer \\ 
  	\hline
  	213.7 & integrate comparison view (additional usage scenario) into chart tab window & Frank Kessler \\ 
  	\hline
  	213.8 & fill comparison/"evolution of a class" view with data & Frank Kessler \\ 
  	\hline
  \end{tabular}
\end{table}


\subsubsection*{Noteworthy Development Aspects}

\subsubsection*{Sprint Review}



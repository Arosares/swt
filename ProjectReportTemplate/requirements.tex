%\flushleft
\emph{(Approx.~6--12~pages of text.) Frank}

\emph{Document and analyse the software's functional requirements, 
non-functional requirements and development constraints. In particular, state 
whether a requirement is derived from the project brief, is an assumption made 
by your group, or has been added by the client. You may apply any documentation 
and analysis technique taught in module SWT-FSE-B or from the 
requirements engineering literature, including techniques based on user stories, 
use cases and prototyping. Properly reference and justify all employed 
techniques.}

\emph{This section shall also include a table containing an overview of all user 
stories. Use Table~\ref{tab:user_stories} as a template, and order the stories 
regarding their ID (story number). Name the source of the story: project brief 
(PB), the client (C), or other sources. You can use the stories' name+ID in the 
sequel to refer to a certain story.}

\begin{table}[!h]
  \caption{List of user stories}
  \centering
  \begin{tabular}{l||l|l|l|l|}
    ID & Name & Size &  Source & Sprint\\
    \hline
    1&Research Apriori&Medium&Project brief&1\\
    1&Testing&Medium&Project brief&5\\
    2&Research StAX Parser&Small&Project brief&1\\
    2&Imports in tree structure&Medium&Assumed by us&2\\ %Original ID was 2.001; 
    3&Testrun chart&Medium&Project brief&3\\
    7&Testrun table&Medium&Project brief&2\\
    8&Association Analysis&Large&Project brief&3\\
    9&Testrun Selection&Medium&Project brief&2\\
    12&Store XML in classes&Medium&Assumed by us&1\\ 
    18&XMLParsing&Medium&Project brief&1\\
   
    
    
    23&Implement Apriori Algorithm&Large&Project brief&4\\
    24&Display Analysis Data&Medium&Project brief&4\\
    25&Additional usage scenario&Medium&Project brief&5\\
    26&Code Documentation&Small&Project brief&5\\
    27&Distribution with suited license&Small&Project brief&5\\ %not sure about origin of 27/28
    28&Creating a help function&Small&Project brief&5\\
    213&Evolution of a class&Large&Assumed by us&4\\ 
  
	\end{tabular}
  \label{tab:user_stories}
\end{table}

As you may notice out user stories have quite strange IDs with a lot of numbers missing in between. This is because in the beginning we wanted to prevent that the IDs indicate a ordering. Retrospectively, we came to the conclusion that this was a mistake, as it caused more confusion in our group.\\
In the end we tried to fix this a little bit (user stories 23-28), however, the project was already quite far in progress, so the majority of user stories still has random IDs.  \\
\newline
In Sprint 0 we scanned the project brief for requirements which we then tried to translate into user stories.\\
In the planning meeting of every sprint we looked into the specific requirements of the feature we wanted to implement to derive more user stories.

\subsection{Requirements derived from project brief}
\subsubsection{Functional requirements}
\begin{itemize}	
	\item A table shall show all tested classes of one testrun ordered by their failure percentage
	\item A chart shall display the change of a specific class over all parsed testruns
	\item Association analysis shall use the well-established Apriori algorithm
	\item XML-files shall be parsed using the StAX parser
	
\end{itemize}

\subsubsection{Non-Functional requirements}
\begin{itemize}
	\item All features shall be tested sufficiently
	\item Source code shall be documented properly
\end{itemize}

\subsubsection{Development constraints}
\begin{itemize}
	\item The software has to run on a standard SWT-lab PC
	\item Software shall be implemented in Java; documented using Jdoc; tested with JUnit;
	\item Copied code shall be cited with its source
	\item Development shall be done via the SWT-Git-repository
	\item Commit messages shall have the following format: \\
	SPRINT n SUBSTANTIAL m \\
	documentation of commit \\
	full name of responsible person
	\item The file structure shall be as follows:
	\begin{itemize}
		\item 'src' containing source files
		\item 'test' containing JUnit tests
		\item 'doc' containing JavaDoc of the sources
		\item 'uml' containing UML-diagram files
	\end{itemize}
	\item Final release shall look as follows:
	\begin{itemize}
		\item source files; unit tests; Jdoc
		\item binary release (jar)
		\item project report
	\end{itemize}
	
\end{itemize}
\subsection{Requirements assumed by us}
\subsubsection{Functional requirements}

\subsubsection{Non-functional requirements}
\begin{itemize}
\item The class overview in the sidebar shall display all classes in a tree structure based on the packages they are in
\end{itemize}
\subsection{Requirements added by client}
\subsubsection{Functional requirements}

\begin{itemize}

%Sprint 1:
\item choose a folder and parse all containing XML files in the folder and its subdirectory.

%Sprint 2:
%Maybe NFR:
\item Table: classes with failure percentages between 50 - 75\% should be highlighted yellow and failure percentages between 75 - 100\% should be highlighted red; it should be possible to hide/show all classes with a failure percentage of 0\%

%Sprint 3:
\item one click in tree sidebars for selecting files (no doubleclick)
\item hover over chart entries shall show testrun information
\item chart shall have its own page instead of being shown in the main window below the class table
\item The functions to filter the Apriori results by confidence and distance should be implemented

\item A second additional usage scenario should be proposed, but due to the given point in time the focus should be to finalize the already implemented functionality.

\end{itemize}

\subsubsection{Non-functional requirements}
\begin{itemize}
%Sprint 2:

\item Failure percentages in table should be shorted to 2 decimal places
%maybeFR
\item Testrun totals: a note, stating that not all totals of the testrun are shown, should be visible; a button should enable the user to see all totals.

\item program shall adapt its layout automatically to lower resolutions or resizings
%Sprint 4:
%NFR
\item The outcome of the Apriori algorithm should be visualized in a better way, since the two tables are not easy to read.
\end{itemize}
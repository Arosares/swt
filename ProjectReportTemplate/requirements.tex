%\flushleft
\emph{(Approx.~6--12~pages of text.) Frank}

\emph{Document and analyse the software's functional requirements, 
non-functional requirements and development constraints. In particular, state 
whether a requirement is derived from the project brief, is an assumption made 
by your group, or has been added by the client. You may apply any documentation 
and analysis technique taught in module SWT-FSE-B or from the 
requirements engineering literature, including techniques based on user stories, 
use cases and prototyping. Properly reference and justify all employed 
techniques.}

\emph{This section shall also include a table containing an overview of all user 
stories. Use Table~\ref{tab:user_stories} as a template, and order the stories 
regarding their ID (story number). Name the source of the story: project brief 
(PB), the client (C), or other sources. You can use the stories' name+ID in the 
sequel to refer to a certain story.}

\begin{table}[!h]
  \caption{List of user stories}
  \centering
  \begin{tabular}{l||l|l|l|l|}
    ID & Name & Size &  Source & Sprint\\
    \hline
    \vdots&\vdots&\vdots&\vdots&\vdots\\
  \end{tabular}
  \label{tab:user_stories}
\end{table}

\subsection{Requirements derived from project brief}

\subsection{Requirements assumed by us}
\subsubsection{Functional requirements}

\subsubsection{Non-functional requirements}
\begin{itemize}
\item The class overview in the sidebar shall display all classes in a tree structure based on the packages they are in
\end{itemize}
\subsection{Requirements added by client}
\subsubsection{Functional requirements}

\begin{itemize}

%Sprint 1:
\item choose a folder and parse all containing XML files in the folder and its subdirectory.

%Sprint 2:
%Maybe NFR:
\item Table: classes with failure percentages between 50 - 75\% should be highlighted yellow and failure percentages between 75 - 100\% should be highlighted red; it should be possible to hide/show all classes with a failure percentage of 0\%

%Sprint 3:
\item one click in tree sidebars for selecting files (no doubleclick)
\item hover over chart entries shall show testrun information
\item chart shall have its own page instead of being shown in the main window below the class table
\item The functions to filter the Apriori results by confidence and distance should be implemented

\item A second additional usage scenario should be proposed, but due to the given point in time the focus should be to finalize the already implemented functionality.

\end{itemize}

\subsubsection{Non-functional requirements}
\begin{itemize}
%Sprint 2:

\item Failure percentages in table should be shorted to 2 decimal places
%maybeFR
\item Testrun totals: a note, stating that not all totals of the testrun are shown, should be visible; a button should enable the user to see all totals.

\item program shall adapt its layout automatically to lower resolutions or resizings
%Sprint 4:
%NFR
\item The outcome of the Apriori algorithm should be visualized in a better way, since the two tables are not easy to read.
\end{itemize}
%\flushleft
\emph{(Approx.~6--12~pages of text.) Frank}

\emph{Document and analyse the software's functional requirements, 
non-functional requirements and development constraints. In particular, state 
whether a requirement is derived from the project brief, is an assumption made 
by your group, or has been added by the client. You may apply any documentation 
and analysis technique taught in module SWT-FSE-B or from the 
requirements engineering literature, including techniques based on user stories, 
use cases and prototyping. Properly reference and justify all employed 
techniques.}

\emph{This section shall also include a table containing an overview of all user 
stories. Use Table~\ref{tab:user_stories} as a template, and order the stories 
regarding their ID (story number). Name the source of the story: project brief 
(PB), the client (C), or other sources. You can use the stories' name+ID in the 
sequel to refer to a certain story.}

\begin{table}[!h]
  \caption{List of user stories}
  \centering
  \begin{tabular}{l||l|l|l|l|}
    ID & Name & Size &  Source & Sprint\\
    \hline
    \vdots&\vdots&\vdots&\vdots&\vdots\\
  \end{tabular}
  \label{tab:user_stories}
\end{table}

%\flushleft
\emph{(Approx.~4--7~pages of text.)}

\emph{Describe and justify the different quality assurance techniques that your group has applied alongside the project's conduct, including the INVEST criteria for the user stories, SMART criteria for the tasks derived from user stories, unit tests for your code, and others.  Illustrate your approach to quality assurance by giving relevant examples for each employed technique. Finally, do not forget to evaluate your software's interfaces (including the GUI).}

To ensure the development stayed on track and produced useable and useful code, our User Stories adhered to the INVEST criteria as much as practical, always detailing a specific feature either wanted by the client or seen as necessary by the Team for the inner workings of the program. 
However, we had to deviate from the template to adjust to the peculiarities of the project at hand:
The limited time of roughly 80 manhours per sprint made it impractical limiting a userstory to "one third of a sprint"; Most often, a single sprint consisted of between two and three user stories, each of which spanned at least half a sprint to implement by between one and four people. 

EXAMPLE: Dafür muss ich echt im Lab gucken. 

When it came to Tasks, our team only gradually realized the practical importance of the "Specific" in the SMART-criteria; our early User Stories frequently only contained one or two Tasks, like "Implement Apriori". This became an apparent problem when we tried to improve our documentation in Sprint 4, and we found progress was very hard to trace when two people worked on a single task for the majority of a sprint. 
Following this revelation, tasks were created based on what part of the code they concerned themselves with, like refactoring a single class, or a single function, for example the implementation of a distance calculation that spanned multiple classes but could still be linked to a single userstory and completed by a single teammember or a team of two. 
This also made the effort required for a single task more easily measurable and allowed us to track our progress in greater detail. 

To ensure the working condition of our software, we employed extensive tests of internal methods wherever applicable. 
This includes all internal classes excepting the controller and model classes that frequently contain too many links to other classes, making testing them alone impractical. Most of the methods in those classes are indirectly tested through the tests of the classes they call upon. 


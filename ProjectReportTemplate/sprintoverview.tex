
\subsection{Sprint Overview}

In this subsection a quick overview of all Sprints is given, including the vision underlying each Sprint. \\
By the end of Sprint 1 TDA should be able to parse the XML files provided by the client into our internal data structure. Furthermore a first paper prototype of TDA should be created to demonstrate our vision of TDA to the client. This way we wanted to make sure, that we understood the project and that he approves of our design decisions. \\ 
In Sprint 2 our goal was to display the information we parsed into our data structure in a GUI. The user should be able to select one or more XML files and the TDA presents a table of one chosen test run, containing all tested classes and their failure percentage. The table is sorted in decreasing order of the failure percantage and the classes with the highest failure percentage are highlighted. \\ 
During Sprint 3 our goal was to display a chart that visualises the evolution of the failure percentage of a class over all loaded test runs. Furthermore we wanted to implement the Apriori algorithm to perform a dependency analysis on failed classes. During this Sprint we received the change request from the client which had to be worked into our system. \\ 
In Sprint 4 we wanted to rework the Apriori algorithm, since our first implementation was not very efficient. We also wanted to display the results of the Apriori algorithm in our GUI. Furthermore the implementation of our first additional usage scenario should be finished. The additional functionality should enable the user to compare a class in 2 different testruns and see how the outcome of its unit tests have changed. \\ 
During Sprint 5 our goal was to finalise our system so that we could present a rounded and finished product to the client. We also wanted to come up with an additional usage scenario for the client, which enables him to further analyse his test runs. \\ 

\subsection{Sprint No.~3}

\emph{(Approx.~2--3~pages of text.) Jan}

\subsubsection*{Sprint Planning}

Goal: Association Analysis and Graph

The goal for Sprint 3 was to finish and merge in the LineChart visual output that missed the Sprint 2 deadline (by then still called "Graph") and the implementation of the Apriori Algorithm. 
For the purpose of creating a working Apriori Implementation, a research story with unknown size was assigned. 
In the first session, everyone not assigned to the LineChart was busy with refactoring previous code or improving the GUI, as we expected additional work to crop up once the chart was integrated into the GUI and work on Apriori could not start until research was done.

Stories: 

\begin{table}[h]
  \caption{Stories and Tasks}
  \label{User Stories Sprint 3}
  \centering
  \begin{tabular}{p{1cm}|p{1cm}|p{5cm}|p{3cm}|p{3cm}|}
  	Nr. & Prio & Story & Tasks & Assigned \\ 
  	\hline
  	\hline
  	R1 & 1 & Research Apriori & none & Jan Martin \\ 
  	\hline
  	8 & 1 & Association Analysis & Implement Apriori & Jan Martin \\ 
  	\hline
  	? & 2 & Testrun graph & ? & Frank Kessler, Simon Meyer \\ 
  	\hline
  	TBD & 3 & WIP & AB & BA \\ 
  	\hline
  \end{tabular}
\end{table}

Stories2: 

\begin{itemize}
	\item Association Analysis (8) 
	\end{itemize}
	
\begin{table}[h]
  \caption{Tasks Story 8}
  \label{Story 8 Tasks}
  \centering
  \begin{tabular}{p{1cm}|p{5cm}|p{3cm}|}
  	Nr & Description & Assigned \\ 
  	\hline
  	1 & Implement Apriori & someone \\ 
  	\hline
  	2 & Something Apriori & someone else \\ 
  	\hline
  \end{tabular}
\end{table}
	
\begin{itemize}	
	\item Research Aprior (R1) 
\end{itemize}

\begin{table}[h]
  \caption{Tasks Story ?}
  \label{Story ? Tasks}
  \centering
  \begin{tabular}{p{1cm}|p{5cm}|p{3cm}|}
  	Nr & Description & Assigned \\ 
  	\hline
  	3 & I don't even & know anymore \\ 
  	\hline
  \end{tabular}
\end{table}

Stories 3:

\paragraph*{User Story 8: Implement Apriori}

\begin{itemize}
	\item Task 1: Implement Apriori  - Assigned to: X
	\item Task 2: Implement Apriori more - Assigned to: Y
\end{itemize}

\subsubsection*{Noteworthy Development Aspects}
\emph{Describe and justify the development approach taken and the artefacts produced in this sprint (e.g., prototypes).  State any peculiarities of this sprint, such as peculiarities  regarding (i)~adopted development practices, (ii)~encountered obstacles, (iii)~questions that arose and needed clarification possibly from the client, or (iv)~important aspects regarding --- or changes to --- your software architecture, your algorithms or your techniques applied to solve a technical problem.}

\subsubsection*{Sprint Review}
\emph{Describe the product increment produced in this sprint. Compare the achieved increment with the sprint goal and the user stories that were chosen for this sprint. Give a brief summary on your group's retrospective, including changes to the product backlog and also to the development process and/or techniques that you installed after the sprint in order to overcome any identified obstacle.}

The original intent for Sprint 3 was to implement both the LineChart visual output that missed the Sprint 2 deadline and the Apriori Algorithm. For this purpose, development time was allotted to research , and some trial and error was expected on both user stories. 
While the GUI implementation of the Chart proved to be easier than expected, the Apriori development did not. Mid Sprint, the Group received the Change Request, and, after analysing the new requirements, it was evident that the planned Apriori implementation would have to be significantly changed after the Change request would be merged in. 
Thus we decided to prioritize the implementation of the treeview necessary for the adapted Apriori both in the program logic and the GUI. 
